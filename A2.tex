% Title:   UofT Art & Sciences Assignment Sample File
% Version: 1.00
% Author:  Kaveh Ghasemloo
% Date:    Sept. 28, 2012
%
% Licence: 
% This work is licensed under the Creative Commons Attribution-ShareAlike 3.0 Unported License. To view a copy of this license, visit http://creativecommons.org/licenses/by-sa/3.0/ or send a letter to Creative Commons, 444 Castro Street, Suite 900, Mountain View, California, 94041, USA.

\documentclass[10pt]{csc_assignment}
\usepackage[]{algorithm2e}
\usepackage{amsmath}
\usepackage{algpseudocode}
\usepackage{qtree}

% ----------------------------------------------------------------
% TODO: Enter the assignment number, your name, and your student number below
% ----------------------------------------------------------------
\AssignmentName{1}
\QuestionCount{5}
\StudentName{John Armstrong, Henry Ku}
\StudentNumber{993114492\textbackslash g2jarmst, 998551348\textbackslash g2kuhenr}

% ----------------------------------------------------------------
\begin{document}


\Acknowledgements{
% ----------------------------------------------------------------
% TODO: Write your acknowledgements below.
% ----------------------------------------------------------------

"We declare that we have not used any outside help in completing this assignment."

% ----------------------------------------------------------------
% Aacknowledgements ends
% ----------------------------------------------------------------
}
\begin{description}

\newpage
\item[Q1.]
% ----------------------------------------------------------------
% TODO: Write your answer to the question below. 
% ----------------------------------------------------------------

\emph{\underline{Algorithm}}\\
\begin{algorithm}[H]
 \LinesNumbered 
 \KwIn{An nxn matrix R such that R[i, j] represents the ratio r$_{ij}$, also R[i, j] = 0 where i = j.}
 \KwOut{\emph{True} if we end up with more of our favourite marbles from some friend x$_{i}$ after initiating a marbling cycle, \emph{False} otherwise.}
 D$^{(0)} = R$\;
 \For{k: 1 ... n}{
 Let D$^{(k)}$ = d$_{ij}^{(k)}$ be an nxn matrix\;
 \For{i: 1 ... n}{
 \For{j: 1 ... n}{
 \If{i = k}{
 d$_{ij}^{(k)}$ = d$_{ij}^{(k-1)}$
 }
 \Else{
 d$_{ij}^{(k)}$ = max(d$_{ij}^{(k-1)}$, d$_{ik}^{(k-1)}$ * d$_{kj}^{(k-1)}$)\;
 }
 \If{d$_{ij}^{(k)}$ * r$_{ji}$ \textgreater ~1}{
 return True\;
 }
 }
 }
 }
 return False\;
\end{algorithm}
\emph{\underline{Running Time}}\\
\emph{Lines 2-16:} O(n$^{3}$) due to triple nested loop.\\
\emph{All other lines:} O(1).\\
\emph{\underline{Description}}\\
The algorithm is an augmented version of Floyd-Warshall as presented 
in CLRS. Because we do not know which series exchanges will give a 
total product greater than one, it makes sense to use an all-pairs
algorithm. However, instead of looking at each pass whether the
length of the path between a pair of vertices could be made more 
minimal we are looking for the a series of exchanges that connect
two vertices where the product of exchanges can be maximized.\\
Essentially the recursive definition is defined as follows:\\
\[
d_{ij}^{(k)} = 
\begin{cases} 
      \hfill r_{ij}    \hfill & \text{ if $k$ = 0} \\
      \hfill max(d_{ij}^{(k-1)}, d_{ik}^{(k-1)} * d_{kj}^{(k-1)})\hfill & \text{ if $k$ $\geqslant$ 1} \\
  \end{cases}
\]
% ----------------------------------------------------------------
% Answer ends
% ----------------------------------------------------------------

\newpage
\item[Q2.]
% ----------------------------------------------------------------
% TODO: Write your answer to the question below. 
% ----------------------------------------------------------------


% ----------------------------------------------------------------
% Answer ends
% ----------------------------------------------------------------

\newpage
\item[Q3.]
% ----------------------------------------------------------------
% TODO: Write your answer to the question below. 
% ----------------------------------------------------------------


% ----------------------------------------------------------------
% Answer ends
% ----------------------------------------------------------------

\newpage
\item[Q4.]
% ----------------------------------------------------------------
% TODO: Write your answer to the question below. 
% ----------------------------------------------------------------

\emph{\underline{Algorithm}}\\
\begin{algorithm}[H]
 \LinesNumbered 
 \KwIn{A list \emph{A} of n houses and a list \emph{B} of m hospitals, where every house and hospital is represented by a pair (x, y) on the plane, and s and t}
 \KwOut{"Good" if each household has access to ahospital under the restrictions listed in the question, and "Eeek" otherwise.}
 G(V, E); \# Empty graph\
 G.V = A $\cup$ B $\cup$ \{ T, S \} \;
 \For{i: 1 ... m}{
 G.addEdge(B[i], T)\;
 }
 \For{j: 1 ... n}{
 G.addEdge(S, A[j])\;
 }
 \For{i: 1 ... m}{
 \For{j: 1 ... n}{
 \# d(u, v) is a distance function\
 \If{d(A[j], B[i]) $\leqslant$ ~s}{
    G.addEdge(A[j], B[i])\; 
    G.capacity(A[j], B[i]) = 1\;
 }
 }
 }
 \ForEach{edge (u, v) in G.E}{
 (u, v).f = 0\;
 }
 \While{there exists a path p from S to T in the residual network G$_{f}$}{
 capacity(p) = min\{ G.capacity(u, v): (u, v) in p\} \;
 \ForEach{edge (u, v) in p}{
 \If{(u, v) in E}{
 (u, v).f = (u, v).f + capacity(p)\;
 }
 \Else{
 (v, u).f = (v, u).f - capacity(p)\;
 }
 }
 }
 maxFlow = 0\;
 \For{i: 1 ... m}{
    maxFlow += (B[i], T).f\;
 }
 \If{maxFlow == length(HS)}{
 return 'Good'\;
 }
 \Else{
 return 'Eeek!'\;
 }
 
\end{algorithm}


% ----------------------------------------------------------------
% Answer ends
% ----------------------------------------------------------------

\newpage
\item[Q5.]
% ----------------------------------------------------------------                                                                               
% TODO: Write your answer to the question below.                                                                                                 
% ----------------------------------------------------------------                                                                               

SOLUTION

% ----------------------------------------------------------------                                                                               
% Answer ends                                                                                                                                    
% ---------------------------------------------------------------- 


\end{description}
\end{document}
